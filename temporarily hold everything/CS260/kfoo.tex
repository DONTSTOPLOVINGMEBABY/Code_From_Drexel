%
% Kurt Schmidt
% Jan 2018
%
% compile with pdflatex:
%   $  pdflatex source.tex
% pdfTeX 3.14159265-2.6-1.40.16 (TeX Live 2015/Debian)
% kpathsea version 6.2.1
%
% EDITOR:  tabstop=2, cols>=120
%

\documentclass[10pt,letterpaper,oneside]{article}

\usepackage[ascii]{inputenc}
\usepackage{amsmath,amsfonts,amssymb,amsthm}
	\allowdisplaybreaks
\usepackage[dvipsnames]{xcolor}
\usepackage[margin=1in]{geometry}
	\setlength{\parindent}{0em}
	\setlength{\parskip}{1em}
\usepackage{graphicx}
\usepackage{forest}
\usepackage{hyperref}


\usepackage[symbol]{footmisc}

\newtheorem{theorem}{Theorem}

\usepackage[section]{algorithm}
%\usepackage{algorithm2e}
\usepackage{algpseudocode}
%\usepackage[noend]{algpseudocode}

\usepackage{array}

\makeatletter
	\newcommand{\@minipagerestore}{\setlength{\parskip}{\medskipamount}}
\makeatother

%%%%%%%%   TIKZ   %%%%%%%%%%%%%%%%%%%
% Set up for B-Trees

\usepackage{tikz}
\usetikzlibrary{calc,shapes.multipart,chains,arrows,positioning}
%\setlength{\tabcolsep}{2pt}

\tikzset{
	level distance = 30pt,
	%every node/.style = {inner sep=0pt, font=\small},
	bad/.style={fill=red!50},
	level 1/.style={sibling distance=120pt},
	level 2/.style={sibling distance=50pt},
	%font=\tt,
	%>= stealth,
	%every picture/.style={thick},
	%pointer/.style={*->},
%	node/.style={
%		align=center,
%		draw,
%		anchor=center,
%	}
}

\newcommand{\twonode}[2]{%
	\begin{tabular}{|c|c|}
		\hline #1 & #2\\ \hline
	\end{tabular}
}

%%%% user definitions %%%%%%%%%%%%%%%%%%%%%%%

	% Some colors
\colorlet{NoteBg}{orange!30}

%\renewcommand{\thefootnote}{\alph{footnote}}

\newcommand{\Problem}[1]{\subsection*{Problem #1}}
\newcommand{\Part}[1]{\subsubsection*{Part #1}}
%\newcommand{\Solution}{\subsubsection*{Solution}}
\newenvironment{Solution}
{
	\color{blue!80!black}
	\subsubsection*{Solution}
}
{}

\newenvironment{ohashtable}[1][]
	{\begin{tabular}[#1]{
		@{} 
		> {\small} r <{\normalsize~\rlap{\fbox{\strut~~}}$~~\rightarrow$~}
		@{} l @{}}}
	{\end{tabular}}


\newcommand\Remark[1]{\colorbox{NoteBg}{\parbox[t][][c]{\textwidth}{#1}}}

%\makeatletter\newenvironment{Remark}
%{
%	\colorbox{NoteBg}%
%		{\parbox[t][][c]{\textwidth}
%{
%	}}
%}\makeatother


\newcommand{\func}[1]{\textsc{#1}}

	% Forms for Big-Oh notation
\DeclareMathOperator{\Omicron}{O}
\DeclareMathOperator{\omicron}{o}

\newcommand{\BigOh}[1]{\Omicron(#1)}
\newcommand{\LittleOh}[1]{\omicron(#1)}
\newcommand{\BigOmega}[1]{\Omega(#1)}
\newcommand{\LittleOmega}[1]{\omega(#1)}
\newcommand{\BigTheta}[1]{\Theta(#1)}

	% Operators for dominance notation
\newcommand{\domeq}{\sim}
\newcommand{\domle}{\preceq}
\newcommand{\domlt}{\prec}
\newcommand{\domge}{\succeq}
\newcommand{\domgt}{\succ}

	% Some convenient definitions for coding
\newcommand{\NOT}{~\textbf{not}~}
\newcommand{\AND}{~\textbf{and}~}
\newcommand{\OR}{~\textbf{or}~}
\newcommand{\TRUE}{\textsc{True}}
\newcommand{\FALSE}{\textsc{False}}

%\DeclareMathOperator{\max}{max}

%%%%%%%%  You edit stuff below this line  %%%%%%%%%%%%%%%%%%%%%%%%%%


\title{Sample Homework in \LaTeX}
\author{Kurt Schmidt}
%\date{23 SEPT 2002}  % uses today, by default

\begin{document}

\maketitle

\Problem{1}

	Description of the problem.

	Note, I changed Solution to an environment, so you can modify the attributes of text in a solution separately.

	\begin{Solution}
		This is a simple paragraph.

		Two linefeeds in a row make a new paragraph.  We can inline math:  Let
		$f_n = 3n^2 + 2n - 17$

		We can put math in its own block:

		\begin{align*}
			\text{Let $n = 5$.  Substituting:} \\
			f(5) = 3(5)^2 + 2(5) - 17 \\
			f(5) = 3*25 + 10 - 17 \\
			\text{So, of course:} \\
			f(5) = 68 
		\end{align*}

		We can get the equals signs to line up:

		\begin{align*}
			\text{Let $n = 5$.  Substituting:} \\
			f(5) & = 3(5)^2 + 2(5) - 17 \\
					& = 3*25 + 10 - 17 \\
			\text{And, here's text:} \\
			f(5) & = 68 
		\end{align*}

		The \textbf{align} environment from \textbf{amsmath} is, apparently, preferred to \textbf{eqnarray}, which
		seems to be buggy.  Read \href{http://www.tug.org/pracjourn/2006-4/madsen/madsen.pdf}{this discussion from someone
		much better than I at \LaTeX}.

		The \textbf{split} environment allows for slightly nice continued lines, I believe, w/appropriate numbering.  Also,
		note the \textbf{operatorname} command for log-like operators which aren't already defined.

		\begin{align}
			\begin{split}
			f_i(x) = & i^2 \operatorname{This} x + i \operatorname{is\_a\_really} x \\
							& + \operatorname{long\_line} x 
			\end{split} \\
			g_i(x) = & i \operatorname{not\_as\_long} x
		\end{align}

	\end{Solution}

\Problem{2}

	\Part{a}
	Here are some sums you'd better have stuck in your head

	\begin{Solution}
		\begin{align}
			\text{Standard identities:} \nonumber \\
			\sum_{i=a}^b r  & = (b-a+1)r \\
			\sum_{i} c(f_i) & = c\sum_{i} (f_i) \\
			\text{(And we have big brackets):} \nonumber \\
			\sum_{i} (f_i+g_i) & = \left(\sum_{i} f_i + \sum_{i} g_i\right) \\
			\text{Closed forms for some common sums:} \nonumber \\
			\sum_{i=1}^m i  & = \frac{m(m+1)}{2} \\
			\sum_{i=1}^m i^2  & = \frac{m(m+1)(2m+1)}{6} \\
			\sum_{i=0}^m ar^i  & = a\frac{r^{m+1}-1}{r-1} \;,r\neq1 \\
			\sum_{i=0}^\infty ar^i  & = \frac{a}{1-r} \;,0<|r|<1 
		\end{align}
	\end{Solution}

	\Part{b}
	Here are some logs you'd better have stuck in your head

	\begin{Solution}
		\begin{align}
			\log_b 1 & = 0 \\
			\log_b b & = 1 \\
			\log_b (xy) & = \log_b x + \log_b y \\
			\log_b\frac{x}{y} & = \log_b x - \log_b y \\
			\log_b x^n & = n\log_b x \\
			\log_b x & = \frac{\ln{x}}{\ln{b}} 
		\end{align}
	\end{Solution}

\Problem{3}
	Here's a definition of Fibonacci numbers

	\begin{Solution}
		\[
			F_n = \left\lbrace%
			\begin{array}{cc}
				F_{n-1} + F_{n-2} & \quad \mbox{if $n>1$} \\
				1 & \quad \mbox{if $n \in{} \{0,1\}$} \\
			\end{array} \right.
		\]
	\end{Solution}

\Problem{4}
	And tables are pretty easy.  \& separates columns, and
	\textbackslash{}\textbackslash{} is a newline in \LaTeX

	\begin{Solution}
		\begin{center}
		\begin{tabular}{ c | r | l }
			Name & $n$ & $(3/2)^n$ \\
			\hline
			Picard & 0  & 1 \\
			Riker & 1  & 1.5 \\
			Worf & 2	& 2.25 \\
			Troi & 3	& 3.375 \\
			Crusher & 4	& 5.0625 \\
			LaForge & 5	& 7.59375 \\
			O'Brien & 6	& 11.390625 \\
			Guinan & 7	& 17.0859375 \\
			Q & 8	& 25.62890625 
		\end{tabular}
		\end{center}

		We can get the decimals to line up:

		\begin{center}
		\begin{tabular}{ c | r | r @{.} l }
			Name & $n$ & $(3/2)^n$ \\
			\hline
			Picard & 0  & 1 & \\
			Riker & 1  & 1 & 5 \\
			Worf & 2	& 2 & 25 \\
			Troi & 3	& 3 & 375 \\
			Crusher & 4	& 5 & 0625 \\
			LaForge & 5	& 7 & 59375 \\
			O'Brien & 6	& 11 & 390625 \\
			Guinan & 7	& 17 & 0859375 \\
			Q & 8	& 25 & 62890625 
		\end{tabular}
		\end{center}
	\end{Solution}

\Problem{5}
	Show the following statements:

	\Part{a}
		$ 3n+7 \in \BigOh{n^2} $
	
	\begin{Solution}
		By the definition, we need to find a $c>0$ and $n_0>0$ such that $cn^2
		\geq 3n+7 , \forall n>n_0$:

		\[
		\begin{array}{lll}
			3n+7 & \leq 3n^2 + 7n^2 & ,\forall n\geq1 \\
			     & =    10n^2 \\
			\text{So, we have} \\
			10n^2 & \geq 3n+7 & ,\forall n>1
		\end{array}
		\]

		We have our 2 witnesses.  $c=10 , n_0=1$
	\end{Solution}
 

\Problem{6}
	Write some nonsense code, by way of example.

	\begin{Solution}
	Here's a list of common math symbols in $LaTeX$:
	\url{https://oeis.org/wiki/List\_of\_LaTeX\_mathematical\_symbols}

	\begin{algorithm}[H]  % Optional float wrapper for algorithmic, algorithmicx
		\caption{Totally optional}

		\begin{algorithmic} % [1] if followed by an arg, line numbering starts there
			\Function{Foo}{$G$:~graph, $c$:~color}
				\State $S \gets$ \textsc{Set}()
				\State $rv \gets 13$
				\State \Comment{This is a comment -- $\BigOh{V^2}$}
				\ForAll{$v \in G$}
					\State $color[v] \gets white$
				\EndFor

				\State \Call{bar}{$Q, c$}

				\item[] % blank line, not numbered
				\While{ \NOT \Call{IsEmpty}{$S$} \AND \Call{Magic}{$v$} } \If{ $x \leq \infty$ }
						\State $\pi[w] \gets v$
						\State \Call{Insert}{ $S, v$ }
					\Else
						\State $\pi[w] \gets u$
					\EndIf
					\State \textsc{Juggle\_Magic}()
				\EndWhile
				\item[]
				\Return $S$
			\EndFunction

			\State  % blank line, numbered with its neighbors
			\For{ $i \gets 1..n$ }
				\For{ $j \gets 1..i$ }
					\State \Call{Print}{ i, j }
				\EndFor
			\EndFor
		\end{algorithmic}
	\end{algorithm}
	\end{Solution}

\end{document}
